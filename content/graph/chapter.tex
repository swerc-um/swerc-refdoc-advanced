\chapter{Graph}

\section{Fundamentals}
    \textbf{Bellman-Ford}\\
    $d_k[v] =$ shortest length from source to $v$ using at most $k$ edges
    \[
    d(v) = \min_u d(u) + w_{uv}
    \]
    \textbf{Floyd-Warshall}\\
    $d_k[u][v] =$ shortest length between $u$ and $v$ using only nodes $< k$
    \[
    d(u, v) = \min_w d(u, w) + d(w, v)
    \]
    \textbf{System of weighted contraints}\\
    $A - B \leq w$: add edge $B \to A$ with weight $w$.  
    Add edge $(0, \text{vtx}, 0)$ for each vertex.  
    Negative cycle: no solution.  
    Otherwise, solution: $-\min(D)$.
    \kactlimport{Dijkstra.py}
    \kactlimport{BellmanFord.py}
    \kactlimport{Dials.h}

    \textbf{Number of Spanning Trees}\\
    % I.e. matrix-tree theorem.
    % Source: https://en.wikipedia.org/wiki/Kirchhoff%27s_theorem
    % Test: stress-tests/graph/matrix-tree.cpp
    Create an $N\times N$ matrix \texttt{mat}, and for each edge $a \rightarrow b \in G$, do
    \texttt{mat[a][b]--, mat[b][b]++} (and \texttt{mat[b][a]--, mat[a][a]++} if $G$ is undirected).
    Remove the $i$th row and column and take the determinant; this yields the number of directed spanning trees rooted at $i$
    (if $G$ is undirected, remove any row/column).

    \textbf{Erdős–Gallai theorem}\\
    % Source: https://en.wikipedia.org/wiki/Erd%C5%91s%E2%80%93Gallai_theorem
    % Test: stress-tests/graph/erdos-gallai.cpp
    A simple graph with node degrees $d_1 \ge \dots \ge d_n$ exists iff $d_1 + \dots + d_n$ is even and for every $k = 1\dots n$,
    \[ \sum _{i=1}^{k}d_{i}\leq k(k-1)+\sum _{i=k+1}^{n}\min(d_{i},k). \]

\section{Network flow}
    \kactlimport{dinic.h}
    \kactlimport{dinic.py}
    \kactlimport{GlobalMinCut.h}
    \kactlimport{GomoryHu.h}
	\kactlimport{MinCostMaxFlow.h}
	\kactlimport{CirculationDemands.py}

    \textbf{Flow with demands}\\
    We make the following changes in the network. We add a new source $s'$ and a new sink $t'$, a new edge from the source $s'$ to every other vertex, a new edge for every vertex to the sink $t'$, and one edge from $t$ to $s$. Additionally, we define the new capacity function $c'$ as:\newline
    $c'((s', v)) = \sum_{u \in V} d((u, v)) \quad \text{for each edge } (s', v),$\newline
    $c'((v, t')) = \sum_{w \in V} d((v, w)) \quad \text{for each edge } (v, t'),$\newline
    $c'((u, v)) = c((u, v)) - d((u, v)) \quad \text{for each edge } (u, v) \text{ in the old network},$\newline
    $c'((t, s)) = \infty.$\newline
    If the new network has a saturating flow (a flow where each edge outgoing from $s'$ is completely filled, which is equivalent to every edge incoming to $t'$ being completely filled), then the network with demands has a valid flow, and the actual flow can be easily reconstructed from the new network. Otherwise, there doesn't exist a flow that satisfies all conditions. Since a saturating flow has to be a maximum flow, it can be found by any maximum flow algorithm.\newline

    \textbf{$s-t$ coupe minimum pour graphe planaire}\\
    Dans le graphe \emph{dual}, chaque cellule est un sommet, et il existe deux sommets supplémentaires $s'$ et $t'$. Le sommet $s'$ représente le bord gauche et haut de la grille et le sommet $t'$ le bord droit et bas. Dans ce nouveau graphe, deux sommets sont reliés s'ils sont séparés par une arête du graphe primal. Le poids des arêtes est le même.
    Tout $s'-t'$ chemin dans le graphe dual de longueur $w$ correspond à une $s-t$ coupe dans le graphe primal de même valeur $w$ et vice-versa.

\section{Matching}
	\kactlimport{MCBF.py}
	\kactlimport{VertexCover.py}
	\kactlimport{Dilworth.py}
	\kactlimport{KuhnMunkres.py}
	\kactlimport{GaleShapley.py}

\section{DFS algorithms}
	\kactlimport{SCC.py}
	\kactlimport{BiconnectedComponents.py}
	\kactlimport{2sat.py}
	\kactlimport{EulerTour.py}
    \kactlimport{TopoSort.py}
    \kactlimport{DfsTree.py}

\section{General}
	\kactlimport{ChromaticNumber.py}
	\kactlimport{MIS.py}
	\kactlimport{EdgeColoring.h}
	\kactlimport{MaximalCliques.h}
	\kactlimport{Edmonds.h}
	\kactlimport{PeoChordal.h}
