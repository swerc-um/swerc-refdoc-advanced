\chapter{Combinatorial}

\section{Permutations}
	\subsection{Factorial}
		\kactlimport{factorial.tex}
		\kactlimport{IntPerm.h}

	\subsection{Derangements}
		Permutations of a set such that none of the elements appear in their original position.
		\[ \mkern-2mu D(n) = (n-1)(D(n-1)+D(n-2)) = n D(n-1)+(-1)^n = \left\lfloor\frac{n!}{e}\right\rceil \]

	\subsection{Burnside's lemma}
		Given a group $G$ of symmetries and a set $X$, the number of elements of $X$ \emph{up to symmetry} equals
		 \[ {\frac {1}{|G|}}\sum _{{g\in G}}|X^{g}|, \]
		 where $X^{g}$ are the elements fixed by $g$ ($g.x = x$).

		 If $f(n)$ counts ``configurations'' (of some sort) of length $n$, we can ignore rotational symmetry using $G = \mathbb Z_n$ to get
		 \[ g(n) = \frac 1 n \sum_{k=0}^{n-1}{f(\text{gcd}(n, k))} = \frac 1 n \sum_{k|n}{f(k)\phi(n/k)}. \]

\section{Partitions and subsets}
	\subsection{Partition function}
		Number of ways of writing $n$ as a sum of positive integers, disregarding the order of the summands.
		\[ p(0) = 1,\ p(n) = \sum_{k \in \mathbb Z \setminus \{0\}}{(-1)^{k+1} p(n - k(3k-1) / 2)} \]
		\[ p(n) \sim 0.145 / n \cdot \exp(2.56 \sqrt{n}) \]

		\begin{center}
		\begin{tabular}{c|c@{\ }c@{\ }c@{\ }c@{\ }c@{\ }c@{\ }c@{\ }c@{\ }c@{\ }c@{\ }c@{\ }c@{\ }c}
			$n$    & 0 & 1 & 2 & 3 & 4 & 5 & 6  & 7  & 8  & 9  & 20  & 50  & 100 \\ \hline
			$p(n)$ & 1 & 1 & 2 & 3 & 5 & 7 & 11 & 15 & 22 & 30 & 627 & $\mathtt{\sim}$2e5 & $\mathtt{\sim}$2e8 \\
		\end{tabular}
		\end{center}

        \kactlimport{Partitions.py}

	\subsection{Lucas' Theorem}
		Let $n,m$ be non-negative integers and $p$ a prime. Write $n=n_kp^k+...+n_1p+n_0$ and $m=m_kp^k+...+m_1p+m_0$. Then $\binom{n}{m} \equiv \prod_{i=0}^k\binom{n_i}{m_i} \pmod{p}$.

	\subsection{Binomials}
		\kactlimport{multinomial.h}

\section{General purpose numbers}
	\subsection{Bernoulli numbers}
		EGF of Bernoulli numbers is $B(t)=\frac{t}{e^t-1}$ (FFT-able).
		$B[0,\ldots] = [1, -\frac{1}{2}, \frac{1}{6}, 0, -\frac{1}{30}, 0, \frac{1}{42}, \ldots]$

		Sums of powers:
		\small
		\[ \sum_{i=1}^n n^m = \frac{1}{m+1} \sum_{k=0}^m \binom{m+1}{k} B_k \cdot (n+1)^{m+1-k} \]
		\normalsize

	\subsection{Stirling numbers of the first kind}
		Number of permutations on $n$ items with $k$ cycles.
		\begin{align*}
			&c(n,k) = c(n-1,k-1) + (n-1) c(n-1,k),\ c(0,0) = 1 \\
			&\textstyle \sum_{k=0}^n c(n,k)x^k = x(x+1) \dots (x+n-1)
		\end{align*}
		$c(8,k) = 8, 0, 5040, 13068, 13132, 6769, 1960, 322, 28, 1$ \\
		$c(n,2) = 0, 0, 1, 3, 11, 50, 274, 1764, 13068, 109584, \dots$

	\subsection{Eulerian numbers}
		Number of permutations $\pi \in S_n$ in which exactly $k$ elements are greater than the previous element. $k$ $j$:s s.t. $\pi(j)>\pi(j+1)$, $k+1$ $j$:s s.t. $\pi(j)\geq j$, $k$ $j$:s s.t. $\pi(j)>j$.
		$$E(n,k) = (n-k)E(n-1,k-1) + (k+1)E(n-1,k)$$
		$$E(n,0) = E(n,n-1) = 1$$
		$$E(n,k) = \sum_{j=0}^k(-1)^j\binom{n+1}{j}(k+1-j)^n$$

	\subsection{Stirling numbers of the second kind}
		Partitions of $n$ distinct elements into exactly $k$ groups.
		$$S(n,k) = S(n-1,k-1) + k S(n-1,k)$$
		$$S(n,1) = S(n,n) = 1$$
		$$S(n,k) = \frac{1}{k!}\sum_{j=0}^k (-1)^{k-j}\binom{k}{j}j^n$$

	\subsection{Bell numbers}
		Total number of partitions of $n$ distinct elements. $B(n) =$
		$1, 1, 2, 5, 15, 52, 203, 877, 4140, 21147, \dots$. For $p$ prime,
		\[ B(p^m+n)\equiv mB(n)+B(n+1) \pmod{p} \]

	\subsection{Labeled unrooted trees}
		\# on $n$ vertices: $n^{n-2}$ \\
		\# on $k$ existing trees of size $n_i$: $n_1n_2\cdots n_k n^{k-2}$ \\
		\# with degrees $d_i$: $(n-2)! / ((d_1-1)! \cdots (d_n-1)!)$

	\subsection{Catalan numbers}
		\[ C_n=\frac{1}{n+1}\binom{2n}{n}= \binom{2n}{n}-\binom{2n}{n+1} = \frac{(2n)!}{(n+1)!n!} \]
		\[ C_0=1,\ C_{n+1} = \frac{2(2n+1)}{n+2}C_n,\ C_{n+1}=\sum C_iC_{n-i} \]
		${C_n = 1, 1, 2, 5, 14, 42, 132, 429, 1430, 4862, 16796, 58786, \dots}$
		\begin{itemize}[noitemsep]
			\item sub-diagonal monotone paths in an $n\times n$ grid.
			\item strings with $n$ pairs of parenthesis, correctly nested.
			\item binary trees with with $n+1$ leaves (0 or 2 children).
			\item ordered trees with $n+1$ vertices.
			\item ways a convex polygon with $n+2$ sides can be cut into triangles by connecting vertices with straight lines.
			\item permutations of $[n]$ with no 3-term increasing subseq.
		\end{itemize}
        \kactlimport{IntPerm.h}

\section{Computation}
    \subsection{Catalan Balanced Sequences}
    \textbf{Function:} \texttt{catalan\_bal} \\
    Counts bracket sequences of length $n$ with balance $\geq 0$. \\
    \textbf{Formula:}
    \[
    \text{catalan\_bal}(n, s, e) =
    \begin{cases}
    0, & (n+s+e)\!\mod\!2 \neq 0 \text{ or } s<0 \text{ or } e<0, \\
    \binom{n}{\frac{n+e-s}{2}} - \binom{n}{\frac{n-e-s-2}{2}}, & \text{otherwise}.
    \end{cases}
    \]

    \subsection{Grid Path Calculations}
    \textbf{Grid Path:} \texttt{path} \\
    Paths from $(0, 0)$ to $(x, y)$:
    \[
    \text{path}(x, y) = \binom{x+y}{x}.
    \]

    \textbf{Avoiding Low Barrier:} \texttt{path\_low} \\
    Avoiding $y=x+b$, $b<0$:
    \[
    \text{path\_low}(x, y, b) = 
    \begin{cases}
    0, & b\geq 0, \\
    \text{path}(x, y) - \text{path}(y-b, x+b), & b<0.
    \end{cases}
    \]

    \textbf{Avoiding Upper Barrier:} \texttt{path\_up} \\
    Avoiding $y=x+b$, $b>0$:
    \[
    \text{path\_up}(x, y, b) = 
    \begin{cases}
    0, & b\leq 0, \\
    \text{path}(x, y) - \text{path}(y-b, x+b), & b>0.
    \end{cases}
    \]

    \textbf{Alternating Constraints:} \texttt{calc\_LUL} \\
    Paths from $(0, 0)$ to $(x, y)$ alternating constraints $y=x+b_1$ and $y=x+b_2$:
    \[
    \text{calc\_LUL}(x, y, b_1, b_2) = 
    \begin{cases}
    0, & x<b_1 \text{ or } y<-b_1, \\
    \text{path}(x-b_1, y+b_1) - \text{calc\_LUL}(y+b_1, x-b_1, -b_2, -b_1), & \text{otherwise}.
    \end{cases}
    \]

    \textbf{Two Barriers:} \texttt{path\_2} \\
    Avoiding $y=x+b_1$ and $y=x+b_2$:
    \[
    \text{path\_2}(x, y, b_1, b_2) = \text{path}(x, y) - \newline
    \text{calc\_LUL}(x, y, b_1, b_2) - \text{calc\_LUL}(y, x, -b_2, -b_1).
    \]

    \subsection{Gambler's Ruin Problem}
    \textbf{Right Boundary:} \texttt{gambler\_ruin\_right} \\
    Probability of reaching $R$ starting at $L$:
    \[
    \text{gambler\_ruin\_right}(L, R, p_{\text{right}}) =
    \begin{cases}
    \frac{L}{L+R}, & p_{\text{right}} = 0.5, \\
    1, & p_{\text{right}} = 1, \\
    0, & p_{\text{right}} = 0, \\
    \frac{1 - v^L}{1 - v^{L+R}}, & \text{otherwise, } v = \frac{1-p_{\text{right}}}{p_{\text{right}}}.
    \end{cases}
    \]

    \textbf{Left Boundary:} \texttt{gambler\_ruin\_left} \\
    Probability of reaching $0$:
    \[
    \text{gambler\_ruin\_left}(L, R, p_{\text{left}}) = 1 - \text{gambler\_ruin\_right}(L, R, 1-p_{\text{left}}).
    \]